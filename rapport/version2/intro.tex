\section{Introduction}
Une des notions principales dans l'abstraction que fournissent les systemes d'exploitation concerne les threads et les processus ainsi que leurs exécutions concurrentielles. Pour obtenir une meilleure compréhension de la construction de ces abstractions, ce projet nous invite à implémenter le coeur d'une bibliothèque de thread en espace utilisateur.
Le principe est de créer un enesemble de fonctions similaires à celles des fonctions de la bibliothèque \textit{lpthread}.\\
Dans un premier temps, nous avons été invités à développer ces fonctions de base (création, destruction, ordonnancement ...) en définissant un ensemble d'objectifs appelés \textit{objectifs libres} que nous avons essayé de respecter. Ceci avait pour but de rajouter des fonctionnalités propres à la gestion de nos threads
