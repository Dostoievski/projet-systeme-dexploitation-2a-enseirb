\subsection{Les perfomances :}
\subsubsection{ygdshgfh:}

\subsection{Les applications :}
\subsubsection{ Calcul de la somme de tous les éléments d'un grand tableau par diviser-pour-régner :}
Cette application se trouve dans le fichier \textbf{somme.c}.\\
Le principe est de diviser le tableau pour en calculer la somme plus rapidement. Notre tableau est de taille 1000, on le divise en dix tableaux de taille 100 en lançant un thread sur chacun des tableaux. Les threads calculent la somme de la partie du tableau qui les concerne, puis on retourne la somme totale.\\
Ceci est effectué avec les fonctions de la bibliothèque \textit{pthread} et les fonctions implémentées de la bibliothèque \textit{thread}, chose qui nous permet de comparer le coût du calcul de la somme des deux différentes manières.\\
On remarque alors que le fait d'ulitiser la bibliothèque \textit{thread} est plus rapide puisque l'on peut en parallel calculer les différentes parties du tableau. Concernant la bibliothèque \textit{pthread}, on peut justifier le fait qu'il soit plus lent relativement à l'autre bibliothèque par le fait qu'il ait besoin d'un appel système, c'est-à-dire qu'il va jusqu'au noyau pour faire toutes les opérations.\\
\subsubsection{ Calcul de la suite de Fibonacci :}
Cette application se trouve dans le fichier \textbf{fibo.c}.\\
\subsubsection{Tri rapide d'un très grand tableau :}
Cette application se trouve dans le fichier \textbf{tri\_rapide.c}.\\

