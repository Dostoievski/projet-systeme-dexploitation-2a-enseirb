\section{Les fonctions principales de la bibliothèque}

Toutes les structures de données définies plus haut ont servi à l'implémentation de fonctions principales de notre bibliothèque qui sont les suivantes:\footnote{Pour un détail sur les prototypes se référer au fichier \texttt{thread.h}, nous donnons juste les principes des fonctions} 

\subsection{Fonction \texttt{thread\_self}}
Cette fonction retourne l'identifiant du thread courant. Le thread qui s'exécute est stocké dans une variable globale et donc à chaque appel de \texttt{thread\_self} on retourne l'identifiant du thread référencé par la variable globale.

\subsection{Fonction \texttt{thread\_create}}
Cette fonction permet de créer un thread. Comme défini plus haut, un thread est représenté par une structure et dans celle-ci se trouve un contexte\footnote{Correspond au champ \textsf{ucontext\_t *context} et un contexte peut être vu comme une suite d'instructions que le thread exécute. On peut même l'assimiler au thread}. A chaque appel de la fonction \texttt{thread\_create}, le contexte du thread appelant est sauvegardé et remplacé par le contexte du thread créé. Ainsi un thread créé est automatiquement lancé et le thread appelant est mis en file d'attente. 

\subsection{Fonction \texttt{thread\_yield}}
À l'appel de cette fonction, le thread appelant entend rendre la main. À l'appel de la fonction \texttt{thread\_yield}, le contexte du thread appelant est sauvegardé et on lance le contexte du prochain thread en file d'attente. Si aucun thread n'attend le thread appelant est automatiquement relancé.

\subsection{Fonction \texttt{thread\_exit}}
À l'appel de cette fonction, le thread appelant est sensé avoir fini son exécution et cherche donc à stocker sa valeur de retour. La fonction \texttt{thread\_exit} se contente de stocker dans la structure du thread appelant sa valeur de retour, de signaler que le thread a terminé en mettant son statut à STATUS\_FINISH. Si le thread était attendu par un autre thread alors le thread attendant est automatiquement relancé, sinon on passe la mais au prochain thread en file d'attente. Si le thread n'était pas attendu et qu'il n'y a aucun thread en file d'attente alors le programme termine.  

\subsection{Fonction \texttt{thread\_join}}
Cette fonction est utilisée pour attendre la terminaison d'un thread. À l'appel de cette fonction, le thread appelant est stoppé jusqu'à ce que le thread qu'il a décidé d'attendre termine. Ainsi la valeur de retour du thread attendu est récupérée et stockée dans une variable passée en paramètre.    
;

\subsection{Les perfomances :}
\subsubsection{Coût de création et d'attente :}
Pour comparer le coût de la création avec thread et pthread, le fichier \textbf{creationAttente.c} a été mis en place.\\
On y crée un thread et un pthread. On a aussi deux fonctions \textit{pthreadfunc} et \textit{threadfunc} qui utilisent respectivement les fonctions des bibliothèques \textit{pthread} et \textit{thread}.\\
Nous remarquons alors que le coût de la création et d'attente avec les fonctions de la bibliothèque \textit{thread} est moindre par rapport aux autres. Nous pouvons expliquer ceci par le fait qu'avec un fonctionnement préemptif (utilisant la bibliothèque \textit{pthread}), le coût des appels système (les threads communiquent directement avec le kernel) s'additionne, contrairement aux fonctions avec la bibliothèque \textit{thread} qui elles, utilisent un ordonnancement coopératif.\\
Il est possible de compiler ce fichier à l'aide de la commande \textit{make creationAttente}.
\subsubsection{Coût du changement de contexte :}
Pour comparer le coût du changement de contexte avec thread et pthread, le fichier \textbf{chgContext.c} a été mis en place.\\
On y crée deux threads et trois pthreads. On a aussi deux fonctions \textit{pthreadfunc} et \textit{threadfunc} qui utilisent respectivement les fonctions des bibliothèques \textit{pthread} et \textit{thread} comme ce qui a été vu précédemment.\\
Nous remarquons alors que le coût dub changement de contexte avec les fonctions de la bibliothèque \textit{pthread} est moindre par rapport aux autres.\\
Il est possible de compiler ce fichier à l'aide de la commande \textit{make chgContexte}.
\subsubsection{Coût de la destuction :}
Pour comparer le coût de la destruction avec thread et pthread, le fichier \textbf{destruction.c} a été mis en place.\\
C'est un fichier semblable au programme exemple qui nous a été fourni, sauf que nous utilisons deux fonctions :  \textit{pthreadfunc} et \textit{threadfunc} qui utilisent respectivement les fonctions des bibliothèques \textit{pthread} et \textit{thread}.\\
Nous remarquons alors que le coût de la destruction avec les fonctions de la bibliothèque \textit{thread} est de loin inférieur à celui avec les autres fonctions.\\
Aussi, nous avons expliciter le fait qu'il soit possible, grâce à la nouvelle implémentation, de créer des threads qui font un \textit{thread\_join} avant le \textit{thread\_exit}, et vis versa. Pour ce, le fichier \textbf{freedMemory.c} a été créé.
Il est possible de compiler ces fichiers à l'aide de la commande \textit{make freedMemory} et \textit{make destuction}.

\subsection{Les applications :}
\subsubsection{ Calcul de la somme de tous les éléments d'un grand tableau par diviser-pour-régner :}
Cette application se trouve dans le fichier \textbf{somme.c}.\\
Le principe est de diviser le tableau pour en calculer la somme plus rapidement. Notre tableau est de taille 1000, on le divise en dix tableaux de taille 100 en lançant un thread sur chacun des tableaux. Les threads calculent la somme de la partie du tableau qui les concerne, puis on retourne la somme totale.\\
Ceci est effectué avec les fonctions de la bibliothèque \textit{pthread} et les fonctions implémentées de la bibliothèque \textit{thread}, chose qui nous permet de comparer le coût du calcul de la somme des deux différentes manières.\\
On remarque alors que le fait d'ulitiser la bibliothèque \textit{thread} est plus rapide puisque l'on peut en parallel calculer les différentes parties du tableau. Concernant la bibliothèque \textit{pthread}, on peut justifier le fait qu'il soit plus lent relativement à l'autre bibliothèque par le fait qu'il ait besoin d'un appel système, c'est-à-dire qu'il va jusqu'au noyau pour faire toutes les opérations.\\
\subsubsection{ Calcul de la suite de Fibonacci :}
Cette application se trouve dans le fichier \textbf{fibo.c}.\\
