\section{Structures de données utilisées}
Pour construire la bibliothèque de \texttt{thread} demandée, on a choisi d'utiliser la structure de donnée \texttt{File} implémentée par une liste doublement chainée pour stocker et manipuler les threads.
\begin{lstlisting}
 Structure fifoThread
|  element * head;  | un pointeur vers le premier élément de la file
|  element * tail;  | un pointeur vers le dernier élement de la file
\end{lstlisting}

La structure \texttt{element} est définie comme ce suit:
\begin{lstlisting}
 Structure element
| thread * th;               | un pointeur vers la stucture thread
| struct element * next;     | un pointeur vers l'élément suivant
| struct element * previous; | un pointueur vers l'élément précédent 
\end{lstlisting}

La structure thread représente toutes les informations concernant un thread:
\begin{lstlisting}
 Structure thread
|  int tid;                 | l'identifiant du thread
|  int priorite;            | la priorité du thread
|  ucontext_t * context;    | le context du thread de tyoe ucontext_t*
|  status state;            | le status du thread qui peut etre soit READY soit FINISH
|  void * valRetour;        | la valeur de retour du thread
|  void ** adresseValRetour;| l'adresse de la valeur de retour du thread
\end{lstlisting}