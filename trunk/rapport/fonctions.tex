\subsection{Les fonctions de la bibliothèque}
Toutes les structures de données définies plus haut on servi à l'implementation de fonctions principales de notre bibliothèque qui sont les suivantes:\footnote{Pour un detail sur les prototypes se referer au fichier \texttt{thread.h}, nous donnons juste les principes des fonctions} 

\subsubsection{Fonction \texttt{thread\_self}}
Cette fonction retourne l'ientifiant du thread courant. Le thread qui s'excute est stocké dans un variable globale et donc à chaque appel de \texttt{thread\_self} on retourne l'identifiant du thread referencé par la variable globale.

\subsubsection{Fonction \texttt{thread\_create}}
Cette fonction permet de créer un thread. Comme defini plus haut, un thread est représenté par une structure et dans celle-ci se trouve un context qui correspond à un code que le thread exécute. A chaque appel de la fonction \texttt{thread\_create}, le context du thread appelant est sauvegardé et remplacer par le context du thread créer. Ainsi un thread créé est-il automatiquement lancé et le thread appelant est mis en file d'attente. 

\subsubsection{Fonction \texttt{thread\_yield}}
À l'appel de cette fonction, le thread appellant entend rendre la main. Lorsqu'un thread est lancé, on garde dans le thread l'information sur le context dans lequel il a été lancé donc \texttt{thread\_yield} sauvegarde dans le thread appellant son context et lance le context dans lequel celui-ci a été precedemment lancé.

\subsubsection{Fonction \texttt{thread\_exit}}
À l'appel de cette fonction, le thread appellant est sensé avoir fini son exécution et cherche donc à stocker sa valeur de retour. La fonction \texttt{thread\_exit} se contente de stocker dans la structure du thread appellant sa valeur de retour, de signaler que le thread à treminé en mettant son statut à STATUS\_FINISH et de lancer le context qui a precédé le sien.

\subsubsection{Fonction \texttt{thread\_join}}
Cette fonction est utilisée pour attendre su'un thread termine. À l'appel de cette fonction, le thread appellant est stoppé jusqu'à ce que le thread qu'il a décidé d'attendre termine. Ainsi sa valeur de retour est-elle recupérée et stockée dans une variable passé en paramètre.    

\subsubsection*{Commentaire}
Il apparait clairement que les fonction \texttt{thread\_exit} et \texttt{thread\_join} soient liées. En effet, un thread est sensé avoir terminé qu'après un appel (et un seul est possible) de la fonction \texttt{thread\_exit}. Un appel \texttt{thread\_join} n'est possible que sur les seuls thread qui font un appel \texttt{thread\_exit} pendant leurs exécution. Cependant l'ordre est sans importance: si on decide de faire un appel \texttt{thread\_join} pour attendre un thread qui a déjà fait un \texttt{thread\_exit} alors on recupère juste sa valeur de retour sinon on reste dans la fonction \texttt{thread\_join} jusqu'à ce que le thread attendu face un \texttt{thread\_exit} pour terminer. On comprend donc par là que c'est dans la fonction \texttt{thread\_join} que la politique d'ordonnancement est mis en oeuvre pendant l'attente d'un thread et c'est elle qui libère les ressources utilisées par les thread qui ontterminé. 
